\documentclass[a4paper,10pt]{article}
\usepackage[utf8]{inputenc}
\usepackage{color}
%opening
\title{}
\author{}

\begin{document}

\maketitle

\begin{abstract}

\end{abstract}

\section{}

\subsection{The EigCG Solver}
The EigCG solver \cite{Stathopoulos:2007zi} solves a  linear system using un-restarted CG and computes a few {\it approximate} eigenvectors with small eigenvalues. 
When the solution for multiple right-hand sides is needed, as is usually the case in lattice calculations, the computed eigenvectors are used to deflate subsequent 
systems leading to a faster convergence. Usually, the small number of eigenvectors computed with EigCG from solving a single linear system is not sufficient to 
produce observed acceleration. More eigenvectors are accumulated {\it incrementally} while solving the first subset of linear systems. The total
number of eigenvectors collected is usually limited by the available storage. In addition, usually one reaches a limit where more eigenvectors doesn't provide a 
substantial acceleration for convergence any more. This phase is called a {\it build-up phase} during which we collect a reasonable number of eigenvectors for deflation. 
After this build-up phase all subsequent systems will be solved with CG together with deflation. Deflation is performed using a single simple projection step. Since 
the computed eigenvectors are approximate, it is usually necessary to restart CG one or more times and apply the deflation step. Combining CG with a deflation projection and restarting
is called {\it init-CG} \cite{Stathopoulos:2007zi}. The algorithm for the entire process is called {\it Incremental EigCG} and is described schematically in Figure \ref{increigcg}.
\subsubsection{Implementation}
The implementation given in tmLQCD doesn't require all the right-hand sides to be available at the beginning but the systems are solved one after the other. Currently, the solver is available only
for the even-odd preconditioned case with or without a clover term. A sample input file is also available. A description of the parameters is given below:

\begin{verbatim}
  Solver              = INCREIGCG  #name of the solver
  UseEvenOdd          = yes #only available option at the moment
  EigCGnrhs           total number of the linear systems to be solved
  EigCGnrhs1          number of the first set of systems that will be 
                      solved to tolerance EigCGtolsq1
  EigCGtolsq1         convergence tolerance for the systems 1,..,EigCGnrhs1
  SolverPrecision     convergence tolerance for systems EigCGnrhs1+1,..,EigCGnrhs
  EigCGnev            number of eigenvectors to be computed by EigCG for every 
                      system in the build-up phase
  EigCGvmax           size of the search subspace used by EigCG 
  EigCGldh            total number of eigenvectors that to be computed
  EigCGrestolsq       tolerance for restarting init-CG 
  EigCGRandGuessOpt   if 0 means use zero initial guess, 1 means use random intial 
                      guess as a volume gaussian spinor
  AddDownPropagator   if yes, the solver will still work but deflation may not be 
                      as efficient since the eigenvectors for +mu and -mu will 
                      be different.
\end{verbatim}

These options allow for some flexibility in setting the tolerance for the linear systems for the first 
set of vectors. This can be helpful if one solves to a higher tolerance when computing eigenvectors but 
then switches to low precision such as the case when using truncated solver methods for example. Using 
a random initial guess can also be useful if one tries to avoid any bias of the computed eigenvectors when using
point sources for example.
\begin{figure}[htbp]
\begin{center}
\fbox{
\begin{minipage}[t]{0.95\textwidth}
\begin{center}
\fbox{Incremental eigCG}
\end{center}
{\bf Description:} Solve the systems $A\ x_i = b_i$ for $i=1,2,\dots,N_r$\\
{\bf Input Parameters:}\\
$n_{ev}:$ Number of eigenvectors to be computed with EigCG.\\
$m :$ Size of the search subspace to be used by EigCG.\\
$N_{ev}:$ Total number of eigenvectors requested (usually $N_{ev}=k*n_{ev}$).\\
$tol:$ tolerance for solving linear systems.\\
$tol_{rest}:$ tolerance for restarting {it init-CG}.\\
{\bf Algorithm:}
\begin{tabbing}
 x\=xxx\=xxx\=xxx\=\kill
 Let $V$ be the matrix whose columns are the $m$ eigenvectors computed by EigCG.\\
 Let $U$ be the matrix whose columns are the accumulated eigenvectors.\\
 Let $H=U^\dagger A U$.\\
 \underline{build-up phase}\\
 \> for $i=1,2,\dots,k$\\
 \> \> $r_0=b_i - A x_0$ where $x_0$ is the initial guess for system $i$.\\
 \> \> Deflate: $x_0' = x_0 + U H^{-1} (U^\dagger r_0).$\\
 \> \> Solve $A x_i = b_i$ using EigCG and $x_0'$ as initial guess and get $n_{ev}$ eigenvectors $V$.\\
 \> \> Append and orthogonalize $V$ to $U$ and update $H$.\\
 \>end for\\
 \underline{init-CG phase}\\
 \> for $i=k+1,\dots,N_r$\\
 \> conv=false\\
 \> \> while (conv=false) \\
 \> \> \> $r_0=b_i - A x_0$ where $x_0$ is the initial guess for system $i$.\\
 \> \> \> Deflate: $x_0' = x_0 + U H^{-1} (U^\dagger r_0).$\\
 \> \> \> Solve $A x_i = b_i$ using CG and $x_0'$ as initial guess to $tol_{rest}$.\\
 \> \> \> Restart: $x_0= x_i$ (the current solution).\\
 \> \> \> If the system converged to $tol$, then conv=true.\\
 \> \> end while\\
 \> end for\\
\end{tabbing}
\end{minipage}
}
\end{center}
\caption{The Incremental EigCG algorithm}
\label{increigcg}
\end{figure}


\begin{thebibliography}{999}
%\cite{Stathopoulos:2007zi}
\bibitem{Stathopoulos:2007zi}
  A.~Stathopoulos and K.~Orginos,
  %``Computing and deflating eigenvalues while solving multiple right hand side linear systems in quantum chromodynamics,''
  SIAM J.\ Sci.\ Comput.\  {\bf 32} (2010) 439
  [arXiv:0707.0131 [hep-lat]].
  %%CITATION = ARXIV:0707.0131;%%
  %44 citations counted in INSPIRE as of 07 Aug 2014
\end{thebibliography}




\end{document}