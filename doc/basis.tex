\subsection{QCD on a lattice}

Quantum Chromodynamics on a hyper-cubic Euclidean space-time lattice
of size $L^3\times T$ with lattice spacing $a$ is formally described
by the action
\begin{equation}
  \label{eq:action}
  S = S_\mathrm{G}[U] + a^4 \sum_x \bar\psi\ D[U]\ \psi
\end{equation}
with $S_\mathrm{G}$ some suitable discretisation of the the Yang-Mills
action $F_{\mu\nu}^2/4$~\cite{Yang:1954ek}. The particular
implementation we are  using can be found below in section 4.2 and
consists of  plaquette and rectangular shaped Wilson loops with
particular  coefficients. $D$ is a discretisation of the Dirac
operator, for which Wilson originally proposed~\cite{Wilson:1974sk} to
use the 
so called Wilson Dirac operator
\begin{equation}
  \label{eq:DW}
  D_W[U] = \frac{1}{2}\left[\gamma_\mu\left(\nabla_\mu +
    \nabla^*_\mu\right) -a\nabla^*_\mu\nabla_\mu \right]
\end{equation}
with $\nabla_\mu$ and $\nabla_\mu^*$
the forward and backward gauge covariant difference operators,
respectively:
\begin{equation}
  \label{eq:covariant}
  \begin{split}
  \nabla_\mu\psi(x) &= \frac{1}{a}\Bigl[U(x,x+a\hat \mu)\psi(x+a \hat \mu) -
  \psi(x)\Bigr]\, , \\    
  \nabla_\mu^* \psi(x) &=
  \frac{1}{a}\Bigl[\psi(x)-U^\dagger(x,x-a\hat\mu)\psi(x-a\hat\mu)\Bigr]\, ,\\
  \end{split}
\end{equation}
where we denote the $\mathrm{SU}(3)$ link variables by $U_{x,\mu}$.
We shall set $a\equiv 1$ in the following for convenience. 
Discretising the theory is by far not a unique procedure. Instead of Wilson's
original formulation one may equally well chose the 
Wilson twisted mass formulation and the corresponding Dirac
operator~\cite{Frezzotti:2000nk}
\begin{equation}
  \label{eq:Dtm}
  D_\mathrm{tm} = (D_W[U] + m_0)\ 1_f + i \mu_q\gamma_5\tau^3
\end{equation}
for a mass degenerate doublet of quarks. We denote by $m_0$ the bare
(Wilson) quark mass, $\mu_q$ is the bare twisted
mass parameter, $\tau^i$ the $i$-th Pauli matrix and $1_f$ the
unit matrix acting in flavour space (see appendix~\ref{sec:gammas} for
our convention). In the framework of Wilson twisted mass QCD only
flavour doublets of quarks can be simulated, however, the two quarks
do not need to be degenerate in mass. The corresponding mass
non-degenerate flavour doublet reads~\cite{Frezzotti:2003xj}
\begin{equation}
  \label{eq:Dh}
  D_h(\bar\mu, \bar\epsilon)  = D_\mathrm{W}\ 1_f +
  i\bar\mu\gamma_5\tau^3 - \bar\epsilon \tau^1 \, .
\end{equation}
It has the property
\[
D_h^\dagger = \tau^1\gamma_5 D_h \gamma_5 \tau^1\,.
\]
Note that this notation is not unique. Equivalently -- as used in
Ref.~\cite{Chiarappa:2006ae} -- one may write
\begin{equation}
  \label{eq:altDh}
  D_h'(\mu_\sigma,\mu_\delta) = D_\mathrm{W}\cdot 1_f +
  i\gamma_5\mu_\sigma\tau^1 + \mu_\delta \tau^3\, ,
\end{equation}
which is related to $D_h$ by $D_h' = (1+i\tau^2)D_h(1-i\tau^2)/2$
and $(\mu_\sigma,\mu_\delta)\to(\bar\mu, -\bar\epsilon)$. 

%%% Local Variables: 
%%% mode: latex
%%% TeX-master: "main"
%%% End: 
